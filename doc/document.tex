\documentclass[12pt,a4paper,openany,oneside]{book}

\usepackage{hyperref}
\usepackage[italian]{babel}


%\usepackage[latin1]{inputenc} %Windows
\usepackage[utf8x]{inputenc} %Linux
%\usepackage[applemac]{inputenc} %Mac

\usepackage{graphicx}
\usepackage[font=small,labelfont=bf,tableposition=top]{caption}

\usepackage{listings}
  \usepackage{courier}
 \lstset{
         language=C++,
         basicstyle=\footnotesize\ttfamily,
         numbers=left,           
         numberstyle=\tiny,       
         numbersep=5pt,        
         tabsize=5,                 
         extendedchars=true,         
         breaklines=true,
         keywordstyle=\textbf,           
         stringstyle=\color{white}\ttfamily,
         showspaces=false,       
         showtabs=false,           
         xleftmargin=17pt,
         framexleftmargin=17pt,
         framexrightmargin=5pt,
         framexbottommargin=4pt,
         showstringspaces=false          
 }
 
 \addto\captionsitalian{%
 	\renewcommand{\lstlistingname}{Codice}}

\setcounter{tocdepth}{3} % fa apparire le subsubsection nell'indice
\setcounter{secnumdepth}{3} % e le numera


\usepackage{amsmath}


\usepackage{framed}

\begin{document}

%Inserisce qui il frontespizio
\begin{titlepage}
\centering 
\includegraphics[width=2.434cm,height=2.565cm]{Images/university_logo.png}

\bigskip

{\Large \textbf{UNIVERSIT\`A DEGLI STUDI DI CATANIA}}

{\scshape
\large
Dipartimento di Matematica e Informatica
}

{\scshape
\normalsize
Corso di Laurea Magistrale in Informatica
}

\bigskip


\hrule


\bigskip


\bigskip


\bigskip


\bigskip

{\itshape
\large
Kevin Speranza
\par}


\bigskip


\bigskip


\bigskip


\bigskip

{\centering
\Large
Benchmark di tecniche di Cloud Removal su dati Sentinel-1 e Sentinel-2
\par}


\bigskip


\bigskip


\bigskip


\bigskip


\bigskip


\bigskip


\begin{minipage}[b]{8 cm}
\hrule

\bigskip

{\centering\scshape 
Relazione Progetto Finale Multimedia
\par}


\bigskip

\hrule
\end{minipage}
\bigskip


\bigskip


\bigskip


\bigskip


\bigskip


\bigskip


\bigskip


\bigskip


\bigskip


\bigskip


\bigskip

{\raggedleft
Professore: Dario Allegra \\
Professore: Filippo Stanco \\
\par}


\bigskip


\bigskip


\bigskip


\bigskip

\hrule

\bigskip

{\centering
Anno Accademico 2025 - 2026
\par}
\end{titlepage}
 %<- richiama il file "parts/titlepage.tex" che contiene il frontespizio

\bibliographystyle{unsrt}
\title{Titolo Della Tesi}
\author{Kevin Speranza}

%Inserisce qui l'abstract
\chapter*{Abstract}
L'abstract va inserito qui.
\cite{iplab}

%Genera l'indice
\tableofcontents

\chapter{Introduzione}
\label{chap:introduction}

I satelliti \textbf{scattano continuamente fotografie} della Terra per controllare lo stato delle foreste, delle città e dei mari. Tuttavia, c'è un grosso problema naturale: le nuvole.
Si stima che, in qualsiasi momento, circa il \textbf{67\% del nostro pianeta sia coperto da nubi}. Questo significa che quando si prova a scattare una foto, l'immagine risulta \textbf{bianca e inutilizzabile}.

\begin{figure}[h]
    \centering
    \begin{subfigure}{0.48\textwidth}
        \centering
        \includegraphics[width=\textwidth]{Images/nuvole_vs_nonuvole1.png}
        \caption{Foto senza nuvole.}
    \end{subfigure}
    \hfill
    \begin{subfigure}{0.48\textwidth}
        \centering
        \includegraphics[width=\textwidth]{Images/nuvole_vs_nonuvole2.png}
        \caption{Foto con nuvole.}
    \end{subfigure}
    \caption{Il problema: a sinistra quello che vorremmo vedere, a destra una foto inutile coperta dalle nuvole}
    \label{fig:cloud_problem}
\end{figure}

\section{Perché è importante rimuoverle?}
Non è solo una questione estetica. Vedere cosa c'è sotto le nuvole è fondamentale per scopi pratici:
\begin{itemize}
    \item \textbf{Disastri Naturali:} Durante alluvioni o uragani il cielo è quasi sempre coperto. Senza una tecnologia che "vede attraverso" le nuvole, i soccorritori non possono sapere quali zone sono allagate.
    \item \textbf{Intelligence:} La \textbf{lotta alla pesca illegale e al contrabbando} dipende dalla capacità di tracciare le navi che spengono i sistemi di identificazione. Senza la rimozione delle nuvole, \textbf{vaste porzioni di oceano diventano "zone cieche"} dove le attività illecite possono proliferare indisturbate.
\end{itemize}

\section{Le Sfide: Perché è difficile?}
Cancellare una nuvola non è semplice per tre motivi principali:
\begin{enumerate}
    \item \textbf{Spessore:} Alcune nuvole sono leggere (si vede un po' sotto), altre sono muri bianchi invalicabili.
    \item \textbf{Ombre:} Le nuvole proiettano ombre scure sul terreno. Anche se togliamo la nuvola, l'ombra nera rimane e falsa i colori della foto.
    \item \textbf{Il Tempo passa:} Il paesaggio cambia (inverno, estate, costruzioni). Non possiamo semplicemente incollare una foto vecchia di un anno fa, perché la realtà a terra potrebbe essere diversa.
\end{enumerate}

\section{Approccio alla Soluzione}

Abbiamo diversi strumenti a disposizione:
\begin{itemize}
    \item \textbf{Tempo:} Sfruttiamo il fatto di avere diverse foto della stessa zona scattate in giorni diversi. Se oggi è nuvoloso, probabilmente la settimana scorsa (o quella successiva) non lo era. Unendo i "pezzi puliti" di giorni diversi, possiamo ricostruire il puzzle completo.
    \item \textbf{Radar:} Integriamo i dati dei satelliti Sentinel-1. A differenza delle macchine fotografiche ottiche, il Radar \textbf{usa onde radio} che attraversano le nuvole senza problemi, \textbf{fornendoci la struttura del terreno sottostante}.
\end{itemize}

\subsection{La Strategia}
Dobbiamo decidere come usare questi dati, ci sono due strade principali:

\begin{itemize}
    \item \textbf{Sequence-to-Point (Seq2Point):} Si usano tutte le $N$ immagini scattate in istanti temporali diversi per concentrare le informazioni su \textbf{un solo scatto finale} (il target).
    \item \textbf{Sequence-to-Sequence (Seq2Seq):} Si cerca di ripulire \textit{tutte} le $N$ immagini della serie.
\end{itemize}

In questo progetto abbiamo scelto la strategia \textbf{Seq2Point} perché è la più pratica ai \textbf{fini dimostrativi} e permette di focalizzarci sulla qualità del risultato finale, senza doverci preoccupare di ricostruire ogni singola immagine della serie.

\section{Obiettivo del Progetto}
L'obiettivo di questa tesi è esplorare lo stato dell'arte delle tecniche di \textit{Cloud Removal} applicate al dataset \textbf{SEN12MS-CR-TS}.

Valuteremo:
\begin{enumerate}
    \item \textbf{Least Cloudy Selection (Baseline 1):} Un metodo di selezione che individua l'acquisizione temporale con la \textbf{minor copertura nuvolosa}.
    \item \textbf{Temporal Mosaicing (Baseline 2):} Una tecnica statistica che sfrutta la \textbf{media temporale} per fondere più acquisizioni e rimuovere gli outlier (nuvole).
    \item \textbf{Deep Learning Multi-modale:} Test tramite inferenza di un approccio ResNet-based \cite{MERANER2020333} che sfrutta la \textbf{fusione di dati Ottici e Radar}.
\end{enumerate}
\input{parts/2_related-works.tex}
\input{parts/3_method.tex}
\chapter{Experiments}

Si affrontano le seguenti fasi:
\begin{itemize}
    \item Creazione Ground Truth
    \item Implementazioni baseline
    \item Inferenza DSen2-CR
\end{itemize}

\subsection{Creazione della Ground Truth}

A causa delle limitazioni dell'hardware locale gli esperimenti sono stati migrati su \textbf{Google Colab}.

Non potendo disporre di un target nativamente esente da nuvole. Come indicano gli autori del paper \cite{sen12mscrts} è necessario l'uso del modello \textit{s2cloudless} \cite{s2cloudless}. Quest'ultimo \textbf{genera una maschera binaria} i cui pixel assumono valori nell'insieme $\{0, 1\}$, \textbf{indicando} rispettivamente le \textbf{aree prive di copertura nuvolosa e quelle coperte da nuvole}.

l'algoritmo ha esaminato tutti i 30 \textit{timestep} di ciascuna porzione spaziale, isolando il riferimento ottimale e salvando i risultati come segue:

\begin{verbatim}
"gt_patches": {
    "1": {
        "reference_timestep": 2,
        "cloud_fraction": 0.0,
        "processing_time": 107.7556
    },
    "2": {
        "reference_timestep": 2,
        "cloud_fraction": 0.0,
        "processing_time": 108.7203
    }
}
\end{verbatim}

I metadati registrati per ciascuna patch descrivono metriche fondamentali per la selezione automatica:
\begin{itemize}
    \item \textbf{\texttt{reference\_timestep}}: timestemp in cui si trova
    \item \textbf{\texttt{cloud\_fraction}}: rappresenta il rapporto di copertura nuvolosa calcolato pixel per pixel dal rilevatore. Un valore pari a $0.0$ certifica l'assenza totale di formazioni nuvolose.
    \item \textbf{\texttt{processing\_time}}: riporta il tempo computazionale (espresso in secondi).
\end{itemize}

Usando la CPU di colab \textbf{ogni patch} ha richiesto circa \textbf{1 minuto} in media per essere processata. Il \textbf{tempo totale} per l'elaborazione di tutte le patch è stato di circa \textbf{5 ore} (300 patch).

\subsection{Metodi}

\subsubsection{Least Cloudy}
L'approccio di base più semplice
\begin{itemize}
    \item Data una patch analizza tutti i giorni a meno di quello GT.
    \item Tra le patch rimanenti, il metodo \textbf{seleziona quella con la minore copertura nuvolosa} usando \textit{s2cloudless} \cite{s2cloudless}.
\end{itemize}

\begin{table}[htbp]
    \centering
    \begin{tabular}{p{0.48\textwidth} p{0.48\textwidth}}
        \textbf{Lati positivi} & \textbf{Lati negativi} \\ \hline
        \begin{itemize}
            \item Estrema semplicità di implementazione.
            \item Costo computazionale molto basso.
            \item Nessuna necessità di addestramento.
            \item I pixel selezionati sono reali e non sintetizzati.
        \end{itemize}
        &
        \begin{itemize}
            \item Forte dipendenza da acquisizioni storiche limpide.
            \item Fallisce se tutte le patch temporali presentano nuvole.
            \item Ignora totalmente i dati radar (S1) e il contesto spaziale.
            \item Rischio di forti incongruenze stagionali rispetto alla GT.
        \end{itemize}
    \end{tabular}
    %\caption{Confronto tra lati positivi e negativi del metodo "Least Cloudy".}
    %\label{tab:least_cloudy_pros_cons}
\end{table}

\begin{figure}[h]
    \centering
    \includegraphics[width=1\textwidth]{Images/least_ex.png}
    \caption{Metodo Least Cloudy}
    \label{fig:leastcloudex}
\end{figure}

\FloatBarrier

\subsubsection{Mosaicing}
Si lavora sul valore del pixel nel tempo:
\begin{itemize}
    \item Se esiste una singola acquisizione senza nuvole nel tempo, il suo valore viene copiato direttamente.
    \item Se esistono più acquisizioni senza nuvole, viene calcolata la media aritmetica dei loro valori.
    \item Se non esiste alcuna acquisizione limpida nella serie temporale, viene assegnato un valore fisso pari a 0.5 come \textit{proxy}. Questo passaggio previene l'inclusione di valori di intensità estremi dovuti alla presenza di nuvole spesse.
\end{itemize}

\begin{table}[htbp]
    \centering
    \renewcommand{\arraystretch}{1.3}
    \begin{tabular}{p{0.45\textwidth} p{0.45\textwidth}}
        \hline
        \textbf{Lati Positivi} & \textbf{Lati Negativi} \\
        \hline
        \textbullet~Sfrutta le informazioni condivise su più frame temporali. & \textbullet~Produce un effetto di sfocatura (\textit{blur}) dovuto all'operazione di media. \\
        \textbullet~Riduce il rumore casuale grazie alla media aritmetica. & \textbullet~Assegna un valore fittizio (0.5) in caso di copertura nuvolosa perenne. \\
        \textbullet~Computazionalmente leggero e non richiede addestramento. & \textbullet~Ignora S1 \\
        \hline
    \end{tabular}
    \caption{Lati positivi e negativi del metodo Mosaicing.}
    \label{tab:mosaicing_pros_cons}
\end{table}

\begin{figure}[h]
    \centering
    \includegraphics[width=1\textwidth]{Images/mosaicingex.png}
    \caption{Metodo Mosaicing}
    \label{fig:mosaicingex}
\end{figure}

\FloatBarrier

\subsubsection{DSen2-CR \cite{MERANER2020333}}

Si usano le \textbf{residual network (ResNet)}. Queste si basano sull'utilizzo di connessioni dirette, chiamate "scorciatoie" (shortcut), \textbf{che saltano alcuni strati computazionali} per trasportare le informazioni verso le parti inferiori della rete.

\begin{figure}[h]
    \centering
    \includegraphics[width=0.5\textwidth]{Images/resnet.png}
    \caption{ResNet: Le connessioni di salto permettono di trasportare l'informazione direttamente all'output, mentre gli strati intermedi imparano a calcolare una correzione additiva (residuo) da applicare all'input.}
    \label{fig:resnet}
\end{figure}

\FloatBarrier

Questo comportamento è cruciale, potremmo avere infatti:
\begin{itemize}
    \item \textbf{Nuvole sottili / velature:} la rete apprende piccole correzioni additive.
    \item \textbf{Cielo sereno:} il long skip trasferisce l'input direttamente.
    \item \textbf{Quasi tutto coperto (nuvole spesse/estese):} il modello sfrutta i segnali SAR e il contesto spaziale per ricostruire pattern strutturali.
\end{itemize}

\FloatBarrier
\begin{figure}[h]
    \centering
    \includegraphics[width=0.8\textwidth]{Images/dsen_arch.png}
    \caption{DSen2-CR model}
    \label{fig:dsenarch}
\end{figure}

\FloatBarrier

Viene usata Cloud-Adaptive Regularized Loss (CARL):
\begin{itemize}
    \item Nelle \textbf{zone nuvolose} si valuta l'errore confrontando la predizione con l'immagine \textbf{target}, in modo da forzare la ricostruzione delle regioni occluse.
    \item Nelle \textbf{zone serene} si valuta l'errore confrontando la predizione con l'immagine di \textbf{input} stessa, preservando i pixel già puliti.
\end{itemize}

Questo approccio costringe la rete a \textbf{non alterare i pixel già puliti}, preservando le informazioni originali ed evitando distorsioni dovute a fusioni multi-temporali.

\FloatBarrier

\vspace{1em}
\subsubsection*{Hardware e Setup di Inferenza}

Non è stato possibile eseguire l'inferenza su servizi cloud come Google Colab. Il \textbf{tempo di compilazione era esaurito}. 

L'esecuzione è stata quindi spostata su hardware locale più potente di quello personale:
\begin{itemize}
    \item CPU Intel Core i7
    \item Scheda video integrata
    \item 32 GB di RAM
\end{itemize} 

Il limite principale è stato proprio la scheda video integrata. Mancando il supporto CUDA, non è stata possibile l'accelerazione hardware.

L'obiettivo iniziale era processare l'intero dataset preso in esame (asiaWest). Tuttavia, la stima dei tempi \textbf{superava le 40 ore}. Il dataset di valutazione è stato quindi \textbf{ridimensionato}. L'analisi è stata limitata alle prime \textbf{50 patch}, per un totale di circa \textbf{1450 immagini}. Questa mole di dati ha comunque richiesto \textbf{oltre 6 ore} di elaborazione parallela.

\vspace{1em}
\subsubsection*{Analisi dei Risultati e Selezione dell'Immagine}

Al termine dell'inferenza, i dati generati sono stati organizzati nella seguente struttura:

\begin{verbatim}
output/
├── 0/
│   ├── 1.tif
│   ├── 2.tif
│   ├── ...
│   └── 30.tif
├── 2/
├── ...
└── 49/
\end{verbatim}

Per ogni patch sono disponibili le relative predizioni per timestep.
È stato applicato l'approccio \textbf{Least Cloudy} per selezionare il candidato migliore.

\begin{figure}[h]
    \centering
    \includegraphics[width=1\textwidth]{Images/resnetex.png}
    \caption{Metodo DSen2-CR}
    \label{fig:resnetex}
\end{figure}

\vspace{1em}
\begin{table}[htbp]
    \centering
    \begin{tabular}{p{0.48\textwidth} p{0.48\textwidth}}
        \textbf{Lati positivi} & \textbf{Lati negativi} \\ \hline
        \begin{itemize}
            \item Elevata robustezza a nuvole ampie e spesse.
            \item I dati radar (S1) ripristinano la struttura del suolo.
            \item La loss adattiva (LCARL) riduce gli artefatti.
            \item Preserva i dettagli originali nelle zone serene.
        \end{itemize}
        &
        \begin{itemize}
            \item Fatica a ricostruire trame urbane complesse.
            \item Non sfrutta al massimo i dati temporali
        \end{itemize}
    \end{tabular}
    %\caption{Confronto tra lati positivi e negativi dell'architettura DSen2-CR.}
    %\label{tab:dsen2_pros_cons}
\end{table}

\FloatBarrier


\subsubsection{SOTA: UnCRtainTS \cite{UnCRtainTS}}

Attualmente questo è il modello state-of-the-art (SOTA) per la task di cloud removal seq2point


\begin{figure}[h]
    \centering
    % INSERISCI QUI L'IMMAGINE: Usa la "Figure 2" del paper, che mostra il diagramma a blocchi del modello (Encoder -> Temporal aggregation -> Decoder).
    \includegraphics[width=1\textwidth]{Images/uncrtaints_model.png}
    \caption{Architettura UnCRtainTS}
    \label{fig:uncrtaints_arch}
\end{figure}

\FloatBarrier
\begin{figure}[h]
    \centering
    % INSERISCI QUI L'IMMAGINE: Usa un ritaglio della "Figure 1" o della "Figure 4", focalizzandoti sulle "Uncertainty maps" colorate in rosso/blu vicino alle immagini reali.
    \includegraphics[width=0.8\textwidth]{Images/uncrtain_ex.png}
    \caption{Immagini esemplari di UnCRtainTS}
    \label{fig:uncrtaints_uncert}
\end{figure}
\FloatBarrier

\vspace{1em}
\begin{table}[htbp]
    \centering
    \begin{tabular}{p{0.48\textwidth} p{0.48\textwidth}}
        \textbf{Lati positivi} & \textbf{Lati negativi} \\ \hline
        \begin{itemize}
            \item Produce mappe di incertezza calibrate che permettono all'utente di scartare o filtrare le ricostruzioni inaffidabili.
            \item Preserva in modo eccellente i dettagli e la struttura spaziale non ricorrendo al downsampling.
            \item Rete molto leggera e parametro-efficiente (solo 0.5M di parametri).
        \end{itemize}
        &
        \begin{itemize}
            \item Oneroso computazionalmente
            \item Sensibile a fenomeni naturali (es. spuma onde della costa)
            \item Difficoltà su serie temporali brevi o singola immagine
        \end{itemize}
    \end{tabular}
    %\caption{Confronto tra lati positivi e negativi dell'architettura UnCRtainTS.}
    %\label{tab:uncrtaints_pros_cons}
\end{table}

\chapter{Results}
I risultati ottenuti sono stati valutati utilizzando le metriche presenti nel paper ovvero:
\begin{itemize}
    \item \textbf{NRMSE (Normalized Root Mean Squares Error) $\downarrow$ :} Misura la differenza media quadratica tra i pixel dell'immagine predetta e quelli dell'immagine reale, normalizzata.
    \begin{itemize}
        \item \textbf{Valore basso:} indica che la predizione è molto accurata e vicina all'originale.
        \item \textbf{Valore alto:} segnala un errore di ricostruzione elevato.
    \end{itemize}
    
    \item \textbf{PSNR (Peak Signal-to-Noise Ratio) $\downarrow$ :}
    \begin{itemize}
        \item \textbf{Valore alto:} indica un'immagine di alta qualità con poco rumore.
        \item \textbf{Valore basso:} denota una forte presenza di distorsioni o artefatti matematici.
    \end{itemize}
    
    \item \textbf{SSIM (Structural Similarity Index) $\uparrow$:} Confronta l'immagine predetta e quella target valutandone la luminanza, il contrasto e la struttura spaziale. Serve a stimare la qualità della ricostruzione in modo simile a come la percepirebbe l'occhio umano.
    \begin{itemize}
        \item \textbf{Valore alto (vicino a 1):} significa che la struttura visiva e i dettagli sono stati preservati fedelmente.
        \item \textbf{Valore basso:} indica la perdita di geometrie e la presenza di sfocature o artefatti visibili.
    \end{itemize}
    
    \item \textbf{SAM (Spectral Angle Mapper):} Calcola l'angolo tra le bande multispettrali di due immagini, trattandole come vettori in uno spazio n-dimensionale. Serve a valutare unicamente la fedeltà delle firme spettrali e dei colori, ignorando le differenze di illuminazione.
    \begin{itemize}
        \item \textbf{Valore basso:} indica un'altissima fedeltà cromatica e spettrale.
        \item \textbf{Valore alto:} significa che i colori o le firme spettrali sono stati alterati.
    \end{itemize}
\end{itemize}

\begin{table}[htbp]
    \centering
    \caption{Confronto delle metriche: Paper vs Subset}
    \label{tab:results_comparison}
    \resizebox{\textwidth}{!}{%
    \begin{tabular}{lcccccccc}
    \hline
    \textbf{Modello} &
    \multicolumn{4}{c}{\textbf{Paper}} &
    \multicolumn{4}{c}{\textbf{Subset}} \\
    \cline{2-5} \cline{6-9}
     & NRMSE ($\downarrow$) & PSNR ($\uparrow$) & SSIM ($\uparrow$) & SAM ($\downarrow$) &
       NRMSE ($\downarrow$) & PSNR ($\uparrow$) & SSIM ($\uparrow$) & SAM ($\downarrow$) \\
    \hline
    Least Cloudy  & 0.0790 & \textbf{31.68} & 0.8150 & 0.213 & \textbf{0.0375} & \textbf{30.90} & \textbf{0.8653} & 0.151 \\
    Mosaicing      & 0.0640 & 26.04 & 0.8110 & 0.250 & 0.0422 & 27.94 & 0.8338 & 0.155 \\
    ResNet         & 0.0600 & 25.42 & 0.8100 & 0.212 & 0.0623 & 25.90 & 0.8518 & \textbf{0.103} \\
    UnCRtainTS     & \textbf{0.0510} & 26.68 & \textbf{0.8360} & \textbf{0.186} &   --   &   --   &   --   &   --   \\
    \hline
    \end{tabular}%
    }
\end{table}

\vspace{0.5cm}

\textbf{Analisi dei Risultati}
\begin{itemize}
    \item \textbf{Incidenza della copertura nuvolosa:} Il dataset globale presenta un'occlusione media vicina al 50\%. Il subset selezionato ha una densità di nuvole inferiore. Al diminuire della percentuale di nuvole, l'errore (NRMSE) si abbassa drasticamente e la similarità strutturale (SSIM) sale in modo fisiologico per tutti i metodi.
    \item \textbf{Omogeneità delle zone campionate (ROI):} L'area presa in esame presenta poca varianza delle località (costa, città, collina). In questi contesti, tecniche euristiche come il Mosaicing riescono a produrre risultati visivamente e metricamente superiori, limitando la generazione di artefatti.
    \item \textbf{Comportamento del Least Cloudy:} Il picco di PSNR ottenuto sul subset (30.90) dall'approccio Least Cloudy è atteso. In scenari con porzioni di cielo limpido o nuvolosità sparsa, mantenere i pixel originali inalterati dell'immagine meno nuvolosa minimizza il rumore matematico assoluto (avvantaggiando il PSNR) rispetto all'uso di reti generative.
    \item \textbf{Fedeltà spettrale della ResNet:} Analizzando il SAM (0.103), emerge chiaramente la validità del modello ResNet. Su questo subset, la rete neurale dimostra una capacità di preservazione delle firme spettrali originali nettamente superiore rispetto alle euristiche di base.
\end{itemize}
\chapter*{Conclusion}

\begin{itemize}
    \item \textbf{Efficienza in condizioni ottimali:} Le euristiche di base, come Least Cloudy e Mosaicing, offrono prestazioni notevoli in termini di riduzione del rumore assoluto (PSNR elevato) quando applicate a scenari con scarsa nuvolosità o elevata omogeneità territoriale. Rappresentano una scelta pragmatica e conveniente esclusivamente se si opera su serie storiche che includono già numerose osservazioni a cielo sereno.
    \item \textbf{Robustezza e fedeltà spettrale:} I modelli basati su reti neurali, come l'approccio ResNet, garantiscono una conservazione nettamente superiore delle firme spettrali e cromatiche originali. Sono necessari per applicazioni a valle (come classificazione o segmentazione) in cui l'accuratezza spettrale è una priorità assoluta.
    \item \textbf{Scelta per scenari complessi:} Quando le singole osservazioni temporali presentano nubi dense e ampie che oscurano completamente il suolo, le soluzioni triviali falliscono in modo critico. In questi casi, è indispensabile l'utilizzo di reti neurali in grado di integrare dati multimodali (SAR-ottico) e multitemporali per ricostruire le informazioni mancanti.
    \item \textbf{Approccio definitivo:} Per un monitoraggio satellitare continuo, robusto e completamente automatizzato, architetture avanzate di deep learning rappresentano la soluzione migliore. Esse superano costantemente le euristiche nella maggior parte delle metriche globali, giustificando l'investimento computazionale con una capacità di inpainting e ricostruzione strutturale che i metodi tradizionali non possono eguagliare.
\end{itemize}

%Inserisce la bibliografia
\newpage
\addcontentsline{toc}{chapter}{Bibliografia}
\bibliography{bibliography}
\end{document}