\chapter{Dataset: SEN12MS-CR-TS\cite{sen12mscrts}}
\label{chap:dataset}

Si tratta di un dataset, curato dai ricercatori della \textbf{Technical University of Munich (TUM)}, progettato specificamente per sviluppare algoritmi di ricostruzione di immagini satellitari e \textbf{rimozione delle nuvole}.

\section{Caratteristiche e Composizione}
Il dataset è \textbf{definito "multi-modale"} in quanto fornisce i seguenti dati:

\begin{itemize}
    \item \textbf{Sentinel-1 (Radar):} Funziona come un radar che attraversa le nuvole. Non scatta una foto a colori, ma \textbf{rileva la forma e la consistenza del terreno}, fornendo un'immagine sempre nitida anche con il brutto tempo.
    \item \textbf{Sentinel-2 (Ottico):} Funziona come una normale fotocamera. Scatta foto a colori molto dettagliate, ma viene bloccato dalle nuvole: se il cielo è coperto, l'immagine mostra solo bianco.
\end{itemize}

\begin{figure}[h]
    \centering
    \includegraphics[width=0.8\textwidth]{Images/sentinel1.png}
    \caption{Funzionamento Sentinel-1}
    \label{fig:radar_penetration}
\end{figure}

Le immagini sono fornite in patch (ritagli) di dimensione $256 \times 256$ pixel, con una risoluzione al suolo di 10 metri per pixel.

Il dataset completo è accessibile pubblicamente tramite i repository della TU Munich:
\begin{itemize}
    \item \textbf{Training Split:} \url{https://mediatum.ub.tum.de/1639953}
    \item \textbf{Test Split:} \url{https://mediatum.ub.tum.de/1659251}
\end{itemize}

\section{Copertura Globale e Struttura Temporale}
Il dataset copre diverse regioni del globo distribuite su tutti i continenti, includendo una \textbf{vasta  variabilità di biomi} (foreste, deserti, aree urbane, ghiacci) \textbf{e stagioni} (inverno, estate, stagioni delle piogge).

\begin{figure}[h]
    \centering
    \includegraphics[width=0.9\textwidth]{Images/dataset_places.png}
    \caption{Distribuzione globale delle patch del dataset SEN12MS-CR-TS. In blu sono presenti le aree relative al training split, mentre in verde quelle del test split.}
    \label{fig:dataset_map}
\end{figure}

\section{Struttura}
La caratteristica distintiva è la \textbf{dimensione temporale}. Per ogni singola patch, il dataset fornisce una sequenza di \textbf{30 acquisizioni} scattate in momenti diversi.

I file seguono la seguente struttura:
\begin{itemize}
    \item \textbf{Dominio del Sensore (\texttt{S1} / \texttt{S2})}: Tipologia di sensore usato
    \item \textbf{Area Geografica (\texttt{ROIs[ID]})}: posizione geografica (Region of Interest)
    \item \textbf{Dimensione Temporale (\texttt{0} -- \texttt{29})}
    \item \textbf{Frammentazione Spaziale (\textit{Patch})}
\end{itemize}

Di seguito una rappresentazione
\begin{verbatim}
    dataset/
    ├── s1_asiaWest_test/ROIs1868/100/S1/
        ├─── 0/   (Cartella del timestep 0)
        │    ├── s1_ROIs1868_100_ImgNo_0_2018-01-03_patch_0.tif
        │    └── ...
        │    └── s1_ROIs1868_100_ImgNo_0_2018-01-03_patch_299.tif
        ├─── .../
        └─── 29
    └── s2_asiaWest_test/ROIs1868/100/S2/
        └── ...
\end{verbatim}

\section{Sottoinsieme di studio: AsiaWest}
Il dataset completo \textbf{occupa diversi Terabyte}. Per questo motivo è stata scelta la ROI \textbf{asiaWest\_test}, in quanto è la più leggera tra quelle disponibili (circa \textbf{24,8 GB}) e permette di lavorare in tempi ragionevoli pur offrendo una buona varietà di paesaggi e nuvole.
\begin{figure}[h]
    \centering
    \begin{subfigure}{0.32\textwidth}
        \centering
        \includegraphics[width=\textwidth]{Images/ex_asiawest1.png}
    \end{subfigure}
    \hfill
    \begin{subfigure}{0.32\textwidth}
        \centering
        \includegraphics[width=\textwidth]{Images/ex_asiawest2.png}
    \end{subfigure}
    \hfill
    \begin{subfigure}{0.32\textwidth}
        \centering
        \includegraphics[width=\textwidth]{Images/ex_asiawest3.png}
    \end{subfigure}
    \caption{Patch presenti in asiawest rilevate con Sentinel-2}
    \label{fig:asiawest_example}
\end{figure}
