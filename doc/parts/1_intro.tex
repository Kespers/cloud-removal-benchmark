\chapter{Introduzione}
\label{chap:introduction}

I satelliti \textbf{scattano continuamente fotografie} della Terra per controllare lo stato delle foreste, delle città e dei mari. Tuttavia, c'è un grosso problema naturale: le nuvole.
Si stima che, in qualsiasi momento, circa il \textbf{67\% del nostro pianeta sia coperto da nubi}. Questo significa che quando si prova a scattare una foto, l'immagine risulta \textbf{bianca e inutilizzabile}.

\begin{figure}[h]
    \centering
    \begin{subfigure}{0.48\textwidth}
        \centering
        \includegraphics[width=\textwidth]{Images/nuvole_vs_nonuvole1.png}
        \caption{Foto senza nuvole.}
    \end{subfigure}
    \hfill
    \begin{subfigure}{0.48\textwidth}
        \centering
        \includegraphics[width=\textwidth]{Images/nuvole_vs_nonuvole2.png}
        \caption{Foto con nuvole.}
    \end{subfigure}
    \caption{Il problema: a sinistra quello che vorremmo vedere, a destra una foto inutile coperta dalle nuvole}
    \label{fig:cloud_problem}
\end{figure}

\section{Perché è importante rimuoverle?}
Non è solo una questione estetica. Vedere cosa c'è sotto le nuvole è fondamentale per scopi pratici:
\begin{itemize}
    \item \textbf{Disastri Naturali:} Durante alluvioni o uragani il cielo è quasi sempre coperto. Senza una tecnologia che "vede attraverso" le nuvole, i soccorritori non possono sapere quali zone sono allagate.
    \item \textbf{Intelligence:} La \textbf{lotta alla pesca illegale e al contrabbando} dipende dalla capacità di tracciare le navi che spengono i sistemi di identificazione. Senza la rimozione delle nuvole, \textbf{vaste porzioni di oceano diventano "zone cieche"} dove le attività illecite possono proliferare indisturbate.
\end{itemize}

\section{Le Sfide: Perché è difficile?}
Cancellare una nuvola non è semplice per tre motivi principali:
\begin{enumerate}
    \item \textbf{Spessore:} Alcune nuvole sono leggere (si vede un po' sotto), altre sono muri bianchi invalicabili.
    \item \textbf{Ombre:} Le nuvole proiettano ombre scure sul terreno. Anche se togliamo la nuvola, l'ombra nera rimane e falsa i colori della foto.
    \item \textbf{Il Tempo passa:} Il paesaggio cambia (inverno, estate, costruzioni). Non possiamo semplicemente incollare una foto vecchia di un anno fa, perché la realtà a terra potrebbe essere diversa.
\end{enumerate}

\section{Approccio alla Soluzione}

Abbiamo diversi strumenti a disposizione:
\begin{itemize}
    \item \textbf{Tempo:} Sfruttiamo il fatto di avere diverse foto della stessa zona scattate in giorni diversi. Se oggi è nuvoloso, probabilmente la settimana scorsa (o quella successiva) non lo era. Unendo i "pezzi puliti" di giorni diversi, possiamo ricostruire il puzzle completo.
    \item \textbf{Radar:} Integriamo i dati dei satelliti Sentinel-1. A differenza delle macchine fotografiche ottiche, il Radar \textbf{usa onde radio} che attraversano le nuvole senza problemi, \textbf{fornendoci la struttura del terreno sottostante}.
\end{itemize}

\subsection{La Strategia}
Dobbiamo decidere come usare questi dati, ci sono due strade principali:

\begin{itemize}
    \item \textbf{Sequence-to-Point (Seq2Point):} Si usano tutte le $N$ immagini scattate in istanti temporali diversi per concentrare le informazioni su \textbf{un solo scatto finale} (il target).
    \item \textbf{Sequence-to-Sequence (Seq2Seq):} Si cerca di ripulire \textit{tutte} le $N$ immagini della serie.
\end{itemize}

In questo progetto abbiamo scelto la strategia \textbf{Seq2Point} perché è la più pratica ai \textbf{fini dimostrativi} e permette di focalizzarci sulla qualità del risultato finale, senza doverci preoccupare di ricostruire ogni singola immagine della serie.

\section{Obiettivo del Progetto}
L'obiettivo di questa tesi è esplorare lo stato dell'arte delle tecniche di \textit{Cloud Removal} applicate al dataset \textbf{SEN12MS-CR-TS}.

Valuteremo:
\begin{enumerate}
    \item \textbf{Least Cloudy Selection (Baseline 1):} Un metodo di selezione che individua l'acquisizione temporale con la \textbf{minor copertura nuvolosa}.
    \item \textbf{Temporal Mosaicing (Baseline 2):} Una tecnica statistica che sfrutta la \textbf{media temporale} per fondere più acquisizioni e rimuovere gli outlier (nuvole).
    \item \textbf{Deep Learning Multi-modale:} Test tramite inferenza di un approccio ResNet-based \cite{MERANER2020333} che sfrutta la \textbf{fusione di dati Ottici e Radar}.
\end{enumerate}