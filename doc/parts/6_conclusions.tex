\chapter*{Conclusion}

\begin{itemize}
    \item \textbf{Scelta in condizioni ottimali:} Le \textbf{euristiche di base}, come Least Cloudy e Mosaicing, offrono prestazioni notevoli quando applicate a scenari con \textbf{scarsa nuvolosità}.
    \item \textbf{Scelta per scenari complessi:} Quando le singole osservazioni temporali presentano nubi dense e ampie che oscurano completamente il suolo, le soluzioni triviali falliscono in modo critico. In questi casi, è indispensabile \textbf{l'utilizzo di reti neurali} in grado di integrare \textbf{dati multimodali} (SAR-ottico) e multitemporali per ricostruire le informazioni mancanti.
    \item \textbf{Approccio definitivo:} Per un monitoraggio satellitare continuo, robusto e completamente automatizzato, \textbf{architetture come DSen2-CR o UnCRtainTS} sono  la soluzione migliore. A parte rari casi è difficile avere sempre condizioni ottimali ed in particolare per il \textbf{task di monitoraggio dei disastri naturali} è l'unica alternativa possibile
\end{itemize}