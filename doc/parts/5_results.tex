\chapter{Results}
I risultati ottenuti sono stati valutati utilizzando le metriche presenti nel paper ovvero:
\begin{itemize}
    \item \textbf{NRMSE (Normalized Root Mean Squares Error) $\downarrow$ :} Misura la differenza media quadratica tra i pixel dell'immagine predetta e quelli dell'immagine reale, normalizzata.
    \begin{itemize}
        \item \textbf{Valore basso:} indica che la predizione è molto accurata e vicina all'originale.
        \item \textbf{Valore alto:} segnala un errore di ricostruzione elevato.
    \end{itemize}
    
    \item \textbf{PSNR (Peak Signal-to-Noise Ratio) $\downarrow$ :}
    \begin{itemize}
        \item \textbf{Valore alto:} indica un'immagine di alta qualità con poco rumore.
        \item \textbf{Valore basso:} denota una forte presenza di distorsioni o artefatti matematici.
    \end{itemize}
    
    \item \textbf{SSIM (Structural Similarity Index) $\uparrow$:} Confronta l'immagine predetta e quella target valutandone la luminanza, il contrasto e la struttura spaziale. Serve a stimare la qualità della ricostruzione in modo simile a come la percepirebbe l'occhio umano.
    \begin{itemize}
        \item \textbf{Valore alto (vicino a 1):} significa che la struttura visiva e i dettagli sono stati preservati fedelmente.
        \item \textbf{Valore basso:} indica la perdita di geometrie e la presenza di sfocature o artefatti visibili.
    \end{itemize}
    
    \item \textbf{SAM (Spectral Angle Mapper):} Calcola l'angolo tra le bande multispettrali di due immagini, trattandole come vettori in uno spazio n-dimensionale. Serve a valutare unicamente la fedeltà delle firme spettrali e dei colori, ignorando le differenze di illuminazione.
    \begin{itemize}
        \item \textbf{Valore basso:} indica un'altissima fedeltà cromatica e spettrale.
        \item \textbf{Valore alto:} significa che i colori o le firme spettrali sono stati alterati.
    \end{itemize}
\end{itemize}

\begin{table}[htbp]
    \centering
    \caption{Confronto delle metriche: Paper vs Subset}
    \label{tab:results_comparison}
    \resizebox{\textwidth}{!}{%
    \begin{tabular}{lcccccccc}
    \hline
    \textbf{Modello} &
    \multicolumn{4}{c}{\textbf{Paper}} &
    \multicolumn{4}{c}{\textbf{Subset}} \\
    \cline{2-5} \cline{6-9}
     & NRMSE ($\downarrow$) & PSNR ($\uparrow$) & SSIM ($\uparrow$) & SAM ($\downarrow$) &
       NRMSE ($\downarrow$) & PSNR ($\uparrow$) & SSIM ($\uparrow$) & SAM ($\downarrow$) \\
    \hline
    Least Cloudy  & 0.0790 & \textbf{31.68} & 0.8150 & 0.213 & \textbf{0.0375} & \textbf{30.90} & \textbf{0.8653} & 0.151 \\
    Mosaicing      & 0.0640 & 26.04 & 0.8110 & 0.250 & 0.0422 & 27.94 & 0.8338 & 0.155 \\
    ResNet         & 0.0600 & 25.42 & 0.8100 & 0.212 & 0.0623 & 25.90 & 0.8518 & \textbf{0.103} \\
    UnCRtainTS     & \textbf{0.0510} & 26.68 & \textbf{0.8360} & \textbf{0.186} &   --   &   --   &   --   &   --   \\
    \hline
    \end{tabular}%
    }
\end{table}

\vspace{0.5cm}

\textbf{Analisi dei Risultati}
\begin{itemize}
    \item \textbf{Incidenza della copertura nuvolosa:} Il dataset globale presenta un'occlusione media vicina al 50\%. Il subset selezionato ha una densità di nuvole inferiore. Al diminuire della percentuale di nuvole, l'errore (NRMSE) si abbassa drasticamente e la similarità strutturale (SSIM) sale in modo fisiologico per tutti i metodi.
    \item \textbf{Omogeneità delle zone campionate (ROI):} L'area presa in esame presenta poca varianza delle località (costa, città, collina). In questi contesti, tecniche euristiche come il Mosaicing riescono a produrre risultati visivamente e metricamente superiori, limitando la generazione di artefatti.
    \item \textbf{Comportamento del Least Cloudy:} Il picco di PSNR ottenuto sul subset (30.90) dall'approccio Least Cloudy è atteso. In scenari con porzioni di cielo limpido o nuvolosità sparsa, mantenere i pixel originali inalterati dell'immagine meno nuvolosa minimizza il rumore matematico assoluto (avvantaggiando il PSNR) rispetto all'uso di reti generative.
    \item \textbf{Fedeltà spettrale della ResNet:} Analizzando il SAM (0.103), emerge chiaramente la validità del modello ResNet. Su questo subset, la rete neurale dimostra una capacità di preservazione delle firme spettrali originali nettamente superiore rispetto alle euristiche di base.
\end{itemize}