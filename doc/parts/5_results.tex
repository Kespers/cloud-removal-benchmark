\chapter{Results}
I risultati ottenuti sono stati valutati utilizzando le metriche presenti nel paper ovvero:
\begin{itemize}
    \item \textbf{NRMSE (Normalized Root Mean Squares Error) $\downarrow$ :} Misura la \textbf{differenza media quadratica tra i pixel} dell'immagine predetta e quelli dell'immagine reale, normalizzata.
    \begin{itemize}
        \item \textbf{Valore basso:} indica che la predizione è molto accurata e vicina all'originale.
        \item \textbf{Valore alto:} segnala un errore di ricostruzione elevato.
    \end{itemize}
    
    \item \textbf{PSNR (Peak Signal-to-Noise Ratio) $\uparrow$ :} Rapporto tra il valore massimo possibile di un pixel e l'errore medio quadratico tra l'immagine predetta e quella reale.
    \begin{itemize}
        \item \textbf{Valore alto:} indica un'immagine di alta qualità con poco rumore.
        \item \textbf{Valore basso:} denota una forte presenza di distorsioni o artefatti
    \end{itemize}
    
    \item \textbf{SSIM (Structural Similarity Index) $\uparrow$:} Confronta l'immagine predetta e quella target valutandone la luminanza, il contrasto e la struttura spaziale. Serve a stimare la \textbf{qualità} della ricostruzione in modo \textbf{simile a come la percepirebbe l'occhio umano}.
    \begin{itemize}
        \item \textbf{Valore alto (vicino a 1):} significa che la struttura visiva e i dettagli sono stati preservati fedelmente.
        \item \textbf{Valore basso:} indica la perdita di geometrie e la presenza di sfocature o artefatti visibili.
    \end{itemize}
    
    \item \textbf{SAM (Spectral Angle Mapper):} Calcola \textbf{l'angolo tra le bande multispettrali} di due immagini, trattandole come vettori in uno spazio n-dimensionale. Serve a valutare unicamente la fedeltà delle firme spettrali e dei colori, ignorando le differenze di illuminazione.
    \begin{itemize}
        \item \textbf{Valore basso:} indica un'altissima fedeltà cromatica e spettrale.
        \item \textbf{Valore alto:} significa che i colori o le firme spettrali sono stati alterati.
    \end{itemize}
\end{itemize}

\begin{table}[htbp]
    \centering
    \caption{Confronto delle metriche: Paper vs Subset}
    \label{tab:results_comparison}
    \resizebox{\textwidth}{!}{%
    \begin{tabular}{lcccccccc}
    \hline
    \textbf{Modello} &
    \multicolumn{4}{c}{\textbf{Paper}} &
    \multicolumn{4}{c}{\textbf{Subset}} \\
    \cline{2-5} \cline{6-9}
     & NRMSE ($\downarrow$) & PSNR ($\uparrow$) & SSIM ($\uparrow$) & SAM ($\downarrow$) &
       NRMSE ($\downarrow$) & PSNR ($\uparrow$) & SSIM ($\uparrow$) & SAM ($\downarrow$) \\
    \hline
    Least Cloudy  & 0.0790 & \textbf{31.68} & 0.8150 & 0.213 & \textbf{0.0375} & \textbf{32.73} & \textbf{0.8653} & 0.151 \\
    Mosaicing      & 0.0640 & 26.04 & 0.8110 & 0.250 & 0.0422 & 27.94 & 0.8338 & 0.155 \\
    ResNet         & 0.0600 & 25.42 & 0.8100 & 0.212 & 0.0623 & 25.90 & 0.8518 & \textbf{0.103} \\
    UnCRtainTS     & \textbf{0.0510} & 26.68 & \textbf{0.8360} & \textbf{0.186} &   --   &   --   &   --   &   --   \\
    \hline
    \end{tabular}%
    }
\end{table}

\vspace{0.5cm}

\textbf{Analisi dei Risultati}
\begin{itemize}
    \item \textbf{Metriche migliori nel subset:} Le prestazioni di ricostruzione dell'immagine \textbf{dipendono fortemente dalla percentuale di copertura nuvolosa}. Ottenere metriche migliori in questo specifico subset è uno scenario sensato, poiché riflette una \textbf{minore percentuale media di copertura nuvolosa} che rende il task di rimozione intrinsecamente meno complesso. Si nota anche dal fatto che \textbf{l'approccio "least cloudy" ottiene risultati eccellenti nel subset}, confermando l'ipotesi.
    \item \textbf{Fedeltà spettrale della ResNet:} Analizzando il SAM (0.103), emerge chiaramente la validità del modello ResNet. Su questo subset, la rete neurale dimostra una \textbf{capacità di preservazione delle firme spettrali originali nettamente superiore} rispetto alle euristiche di base, a conferma del fatto che gli approcci basati sul deep learning superano le soluzioni banali nella ricostruzione delle informazioni.
    \item \textbf{Least cloudy migliore sul PSNR nel paper:} Il fatto che l'euristica "least cloudy" ottenga il PSNR in generaleindica che, in termini di rapporto segnale-rumore valutato sull'intera immagine, copiare i pixel inalterati dell'osservazione meno nuvolosa risulta la \textbf{strategia meno rumorosa}. \textbf{I modelli generativi} o i metodi di media (come il mosaicing), pur riuscendo a ricostruire le aree coperte da nubi, \textbf{tendono a introdurre inevitabili artefatti}, sfocature o alterazioni cromatiche nelle zone già limpide. Di conseguenza, il "least cloudy" massimizza il PSNR \textbf{semplicemente preservando intatta la qualità originale dell'immagine sorgente}.
\end{itemize}